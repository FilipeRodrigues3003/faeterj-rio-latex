\chapter{O Terceiro Capitulo}
 

\section{Citações longas}

\begin{flushright}
\begin{minipage}{0.75\textwidth} % 75% de 160; margem para citações longas
\begin{quotation}
``O importante é não parar de questionar. A curiosidade tem sua própria razão
para existir. Uma pessoa não pode deixar de se sentir reverente ao contemplar
os mistérios da eternidade, da vida, da maravilhosa estrutura da realidade. Basta
que a pessoa tente apenas compreender um pouco mais desse mistério a cada dia.
Nunca perca uma sagrada curiosidade". - Albert Einstein
\end{quotation}
\end{minipage}
\end{flushright}

\section{Tabelas}

\begin{table}[h]
	\centering
		\begin{tabular}{|c|c|c|}
		  \hline
			A & B & A + B \\ \hline
			0 & 0 & 0 \\ \hline
			0 & 1 & 1 \\ \hline
			1 & 0 & 1 \\ \hline
			1 & 1 & 1 \\ \hline
		\end{tabular}
		\caption{Tabela verdade para porta OR.}
\end{table}

\begin{table}[h]
	\centering
		\begin{tabular}{|c|c|c|}
		  \hline
			A & B & $\overline{A + B}$ \\ \hline
			0 & 0 & 1 \\ \hline
			0 & 1 & 0 \\ \hline
			1 & 0 & 0 \\ \hline
			1 & 1 & 0 \\ \hline
		\end{tabular}
		\caption{Tabela verdade para porta NOR.}
\end{table}

É possível criar uma tabela com multicolunas onde uma célula pode ser construída 
com o agrupamento de células vizinhas em uma linha:

\begin{table}[h]
	\centering
	\begin{tabular}{|c|cc|} \hline
		1a linha 1a coluna & \multicolumn{2}{|c|}{2a e 3a colunas mescladas nesta linha}\\ \hline
		2a linha 1a coluna & 2a linha 2a coluna & 2a linha 3a coluna \\
		3a linha 1a coluna & 3a linha 2a coluna & 3a linha 3a coluna \\ \hline
	\end{tabular}
	\caption{Tabela utilizando o comando multicolumn.}
\end{table}


\section{Figuras}

Para inserir figuras deve-se colocar no preâmbulo o pacote graphicx e depois usar o comando
que permite inserir figura.

Por exemplo:
 
  \begin{figure}[h]
    \centering
    \includegraphics[width=4cm, height=6cm, angle=30, scale=0.5]{img/grafico1.jpg}
    \caption{Título da figura.}
    \label{fig:NomeDeReferenciaParaAFigura}
  \end{figure}
  
\begin{verbatim}
  \begin{figure}[h]
    \centering
    \includegraphics[width=4cm, height=6cm, angle=30, scale=0.5]{img/grafico1.jpg}
    \caption{Título da figura.}
    \label{fig:NomeDeReferenciaParaAFigura}
  \end{figure}
\end{verbatim}

No exemplo acima, o arquivo \emph{img/grafico1.jpg} deve estar na pasta \textbf{img} que deverá estar no mesmo diretório do texto. 

Deve ser observado que nenhuma figura foi adicionada ao documento 
para que este template fosse disponibilizado em arquivo único.
