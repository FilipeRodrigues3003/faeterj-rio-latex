\chapter*[Introdução]{Introdução}
\addcontentsline{toc}{chapter}{Introdução}
% ----------------------------------------------------------
A monografia em si é composta por 3 etapas: introdução, desenvolvimento e conclusão. São nessas 3 etapas que a tese será defendida com argumentos lógicos e baseados em dados reais.

A introdução é a parte inicial da sua tese. Nela, os temas de seu trabalho serão mostrados, mas sem muito aprofundamento teórico. É importante não confundir a introdução com o resumo. Eles, até certo ponto, possuem um grau de semelhança, entretanto, a introdução é muito mais aprofundada que o resumo e é escrita em vários parágrafos, sem restrição de número de palavras.

Para a elaboração da introdução é aconselhável a execução por partes. A cada tema pesquisado, escreva o seu correspondente na introdução, pois desse modo quem estiver escrevendo terá muito mais o que falar sobre o tema e o fará com mais precisão do que se fosse escrever sobre todos os temas de uma vez.

Enfim, a introdução é, como o próprio nome diz, a parte introdutória da monografia. Nela, os temas serão apresentados e já pode ser definida a maneira como determinado tema será abordado, desde que não se entre em muitos detalhes acerca do mesmo.


