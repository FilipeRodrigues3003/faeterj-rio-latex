\chapter{O Quarto Capitulo}
% ---


\section{Referências}

Exemplos:

Exemplo de referência simples \cite{ctan}.

Exemplo de referência dupla \cite{ctan} e \cite{signal}.

Exemplo de referência múltipla ($> 2$) \cite{signal}.

\section{Equações}

Existem diversas formas de incluir equações em um documento elaborado em \LaTeX.
Deve-se notar que elas podem ser inseridas no meio do texto como em destaque.

No meio do texto: A equação $a^{2}=b^{2}+c^{2}$ representa...

Em destaque: \[ a^{2}=b^{2}+c^{2} \]

É possível escrever equações que envolvam símbolos utilizados em Cálculo.
A seguir, serão apresentados alguns exemplos.

Série de Fourier na forma complexa:

\begin{equation}
	f(t) = \sum_{n=-\infty}^{\infty} c_{n} \; e^{\frac{i \pi n t}{L}}
\end{equation}

onde:

\begin{equation}
	c_{n} = \frac{1}{2L} \int_{c}^{c+2L} f(t) \, c_{n} \; e^{-\frac{i \pi n t}{L}} \; dt
\end{equation}

\vspace{5mm}

Equações de Maxwell para campos variantes no tempo:

Na forma integral:

\begin{equation}
	\oint \vec{H} \cdot d\vec{L} = \int \Bigg( \vec{J}+\frac{\partial \vec{D}}{\partial t} \Bigg) d\vec{S} 
\end{equation}

\begin{equation}
	\oint \vec{E} \cdot d\vec{L} = - \int \frac{\partial \vec{B}}{\partial t} \; d\vec{S} 
\end{equation}

\begin{equation}
	\int_{S} \vec{D} \cdot d\vec{S} = \int_{V} \rho \; d\vec{V}
\end{equation}

\begin{equation}
	\int_{S} \vec{B} \cdot d\vec{S} = 0
\end{equation}

Na forma diferencial:

\begin{equation}
	\nabla \times \vec{H} = \vec{J}+\frac{\partial \vec{D}}{\partial t}
\end{equation}

\begin{equation}
	\nabla \times \vec{E} = - \frac{\partial \vec{B}}{\partial t}
\end{equation}

\begin{equation}
	\nabla \cdot \vec{D} = \rho
\end{equation}

\begin{equation}
	\nabla \cdot \vec{B} = 0
\end{equation}
