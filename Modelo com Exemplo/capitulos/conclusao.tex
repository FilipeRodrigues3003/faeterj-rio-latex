
\chapter{Considerações Finais}
% ---

A conclusão é a finalização do trabalho textual. Com base nas argumentações dos itens do desenvolvimento, será realizada a conclusão das ideias apresentadas na monografia.

Nesta etapa deve ser usada uma linguagem mais direta, visando à persuasão do leitor. Além disso, é importante evitar períodos muito longos e fazer uso de conectivos para juntar as ideias, a fim de tornar o texto o mais lógico possível. Uma conclusão fraca arrasará sua monografia, pois um trabalho sem um ponto final é um trabalho inacabado. Portanto, muita atenção com a conclusão, pois ela pode decidir se seu trabalho foi bem-sucedido ou não.

Enfim, a conclusão é a convergência das ideias tratadas em toda sua monografia, visto que há um encaminhamento das mesmas, 
através de pensamentos lógicos, para uma definição.
