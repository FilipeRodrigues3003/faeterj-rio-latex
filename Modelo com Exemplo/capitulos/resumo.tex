
% resumo em português
\setlength{\absparsep}{12pt} % ajusta o espaçamento dos parágrafos do resumo
\begin{resumo}

No início dos tempos, Donald E. Knuth criou o \TeX. Algum tempo depois, Leslie Lamport criou o \LaTeX. Graças a eles, não somos obrigados a usar o Word.

 \textbf{Palavras-chave}: \TeX , LaTeX, FAETERJ, Modelo.

\end{resumo}

% resumo em inglês
\begin{resumo}[Abstract]
 \begin{otherlanguage*}{english}
   This is the english abstract.

   \vspace{\onelineskip}

\noindent
  \textbf{Keywords}:  Keyword1, Keyword2, Keyword3, Keyword4.
 \end{otherlanguage*}
\end{resumo}

% resumo em francês
%\begin{resumo}[Résumé]
% \begin{otherlanguage*}{french}
%    Il s'agit d'un résumé en français.

%   \textbf{Mots-clés}: latex. abntex. publication de textes.
% \end{otherlanguage*}
%\end{resumo}

% resumo em espanhol
%\begin{resumo}[Resumen]
% \begin{otherlanguage*}{spanish}
%   Este es el resumen en español.
%
%   \textbf{Palabras clave}: latex. abntex. publicación de textos.
% \end{otherlanguage*}
%\end{resumo}
% ---