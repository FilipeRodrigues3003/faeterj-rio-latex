\chapter{O Segundo Capitulo}
% ---
No ambiente de edição de texto \LaTeX, 
os textos podem ser divididos em fragmentos
(parte \index{parte}, capítulo \index{capítulo!\verb=\=chapter}, 
seção \index{seção}, subseção \index{subseção}, etc.).
Em cada um dos fragmentos de um texto,
o primeiro parágrafo \index{parágrafo} nunca é identado.


Texto do segundo parágrafo.
Continuação do texto.
Continuação do texto.
Continuação do texto.
Continuação do texto.

Texto de um novo parágrafo.
Continuação do texto.
Continuação do texto.
Continuação do texto.
Continuação do texto.


\section{Primeira seção do segundo capítulo}

Texto inicial...
Texto inicial...
Texto inicial...
Texto inicial...
Texto inicial...
Texto inicial...
Texto inicial...
Texto inicial...

\section{Segunda seção do segundo capítulo}

Mais texto... Mais texto...
Mais texto... Mais texto...
Mais texto... Mais texto...
Mais texto... Mais texto...

\section{Terceira seção do segundo capítulo}

Texto ... Texto ...
Texto ... Texto ...
Texto ... Texto ...
Texto ... Texto ...

\section{Quarta seção do segundo capítulo}

Um alerta sobre a divisão do texto um subpartes 
deve ser feito, conforme exemplo abaixo. 

\subsection{Primeira subseção da quarta seção do segundo capítulo}

Texto ... Texto ...
Texto ... Texto ...
Texto ... Texto ...
Texto ... Texto ...

\subsubsection{Primeira subsubseção da primeira subseção 
               da quarta seção do segundo capítulo}

Deve-se notar que uma subsubseção não é mais numerada,
como é feito com o capítulo, com a seção e com a subseção.

No \LaTeX, 
considera-se que a quebra excessiva do texto em subpartes 
confunde o leitor,
pois, além de se ater ao assunto do texto, 
ele ainda é obrigado a manter sua atenção presa à estrutura do texto. 

Ao invés de quebrar o texto em inúmeras subpartes,
recomenda-se que o texto seja melhor construído,
chegando-se, no máximo, na profundidade de uma subsubseção.

Para maiores informações, consulte o site do Comprehensive TeX Archive Network \cite{ctan}.
